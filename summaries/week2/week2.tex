\documentclass[journal, 10pt]{article}

\usepackage{graphicx}
\usepackage{amssymb}
\usepackage{amsmath}
\usepackage{amsthm}

\usepackage{alltt}
\usepackage{float}
\usepackage{color}
\usepackage{url}

\usepackage{balance}
\usepackage[TABBOTCAP, tight]{subfigure}
\usepackage{enumitem}
\usepackage{pstricks, pst-node}

\usepackage{geometry}
\geometry{textheight=8.5in, textwidth=6in}

\newcommand{\cred}[1]{{\color{red}#1}}
\newcommand{\cblue}[1]{{\color{blue}#1}}

\usepackage{hyperref}
\usepackage{geometry}

\def\name{E. Devin Foulger}

\input{pygments.tex}

\hypersetup {
        colorlinks = true,
        urlcolor = black,
        pdfauthor = {\name},
        pdfkeywords = {cs444 "Operating Systems 2"},
        pdfsubject = {CS444 Weekly Summary 1},
        pdfpagemode = UseNone
}

\begin{document}

\title{Week 2 Summary}
\author{\name}
\date{} %So that it won't display date with title

\maketitle

Robert Love, in his book, Linux Kernel Development, explains how to manage and create processes as well as 
explaining how process scheduling works. The author, in chapter three, covers the process structure, process states 
and process creation, while explaining how process scheduling works in chapter four. Love's purpose is to show the 
reader how important processes are and how process scheduling is very important in order to show that scheduling can
greatly affect performance. He sets up an educational tone for his audience which consits of students who are most 
likely computer science and electrical engineers.

\end{document}

