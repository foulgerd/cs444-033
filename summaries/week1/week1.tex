\documentclass[journal, 10pt]{article}

\usepackage{graphicx}
\usepackage{amssymb}
\usepackage{amsmath}
\usepackage{amsthm}

\usepackage{alltt}
\usepackage{float}
\usepackage{color}
\usepackage{url}

\usepackage{balance}
\usepackage[TABBOTCAP, tight]{subfigure}
\usepackage{enumitem}
\usepackage{pstricks, pst-node}

\usepackage{geometry}
\geometry{textheight=8.5in, textwidth=6in}

\newcommand{\cred}[1]{{\color{red}#1}}
\newcommand{\cblue}[1]{{\color{blue}#1}}

\usepackage{hyperref}
\usepackage{geometry}

\def\name{E. Devin Foulger}

\input{pygments.tex}

\hypersetup {
        colorlinks = true,
        urlcolor = black,
        pdfauthor = {\name},
        pdfkeywords = {cs444 "Operating Systems 2"},
        pdfsubject = {CS444 Weekly Summary 1},
        pdfpagemode = UseNone
}

\begin{document}

\title{Week 1 Summary}
\author{\name}
\date{} %So that it won't display date with title

\maketitle

Robert Love, in his book, Linux Kernel Development, explains the history of the Linux kernel and how to set it up. 
The author covers the history of Linux in chapter one and explains how to start using the Linux kernel, in chpater two,
by providing where to find the kernel, how to install it, and its important flags. Love's purpose is to give some
insight on when and why the Linux kernel was made and how to properly set it up, in order to guide the reader in an
attempt to show them the right way of building the kernel. He sets up an educational tone for his audience which 
consits of students who are most likely computer science and electrical engineers.

\end{document}

